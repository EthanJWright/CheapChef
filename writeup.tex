\documentclass[12pt]{article}
\setlength{\oddsidemargin}{0in}
\setlength{\evensidemargin}{0in}
\setlength{\textwidth}{6.5in}
\setlength{\parindent}{0in}
\setlength{\parskip}{\baselineskip}

\usepackage{xcolor}
\usepackage{amsmath,amsfonts,amssymb}
\usepackage{qtree}
\usepackage{enumerate}
\usepackage{multirow}
\usepackage{graphicx}
\usepackage{forest}


\begin{document}

CSCI 4448 Spring 2016 \hfill Project Write-up 1\\

\hrulefill
%%%%%%%%%%%%%%%%%%%%%%%%% SECTION 1  %%%%%%%%%%%%%%%%%%%%%%%%%
\begin{enumerate}
\item\textbf{Team: }
\begin{enumerate}[i.]
\item Ethan Wright
\item Julio Lemos
\item Petar Nguyen
\end{enumerate}
\hrulefill
\item\textbf{Title:} 
CheapChef  \\

\hrulefill
\item\textbf{Description: } \\
A web application that allows the user to input common items that a college student has easy
access to that will then generate a list of possible cheap meals that could be easily created. \\
\hrulefill
\item\textbf{Actors: } \\
\begin{itemize}
\item Customers can select common ingredients from a sorted list on the website
\item Customers select the type of food they would like
\item Customers can select the cost of extra ingredients they do not have to purchase
\item Customers can select the maximum number of ingredients they would like to have to involve in the meal
\item Customers can save and download the recipe that is generated for them
\end{itemize}
\hrulefill
\newpage
\item\textbf{Stretch Functionality: } \\
\begin{itemize}
\item Allow for ingredient list / recipe integration with google keep notes
\item Make a version as an app
\item Create a user sign-in, use the user information to generate customized recipes based on privious accepted recipes. 
\end{itemize}


\end{enumerate}



\end{document}
